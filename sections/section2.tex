\section{システム概要}
弊社が提案する地域特化型SNS「こじゃんとやまっぷ」の利用の流れを図 \ref{fig:sec2/sys} に示す.

\begin{figure}[H]
    \centering
    \includesvg{sections/section2_system.svg}
    \caption{利用の流れ}
    \label{fig:sec2/sys}
\end{figure}
一般会員がシステムにログインすると,投稿や投稿の閲覧を行うための地図表示画面に移動する.投稿を行う場合は利用者は「新規投稿」を選択し,投稿のタイトル・説明・ジャンル・ピンを立てたい場所を入力し,任意で画像を添付する.必要な情報を入力し,「投稿する」を選択すると投稿が公開される.このとき,一般会員の場合は匿名で投稿され,事業者会員の場合は事業者名が表示され会員自身が設定したアイコンがピンに表示される.他の利用者の投稿を閲覧したい場合は投稿一覧か地図上に表示されているピンを選択する.他の利用者の投稿にはリアクションを行いたい場合は投稿を選択し,「リアクション」を選択することで行うことができ,今までにその投稿についたリアクション数を確認することもできる.また,不適切な投稿に対しては通報を行いたい場合は,「通報」を選択し,通報理由を記入し「通報する」を選択することで管理者へ通報が行われる.さらに一般会員は投稿をキーワード検索やジャンル・投稿期間で絞り込むことや,新着順・リアクション数順・距離順に表示を並べ替えることができる.

利用者が自身の情報について確認したい場合は,ホーム画面から「マイページ」を選択すると,登録されているメールアドレス・アカウント種別・登録日を確認することができる.マイページから「投稿履歴」タブを選択することで会員自身が投稿したピンの履歴の閲覧や投稿の削除を行うことができる.「リアクション履歴」タブを選択することで会員自身がリアクションを行なった投稿の履歴を閲覧することができる.「設定」タブからは,会員自身がブロックしたユーザーの一覧やアカウントの削除を行うことができる.一般会員が事業者会員へとなりたい場合は,「マイページ」の「事業者登録を申請」から申請を行うことができる.「事業者登録を申請」を選択し,店舗名・電話番号・住所を入力し「申請する」を選択することで管理者側へ事業者会員へとなりたい旨が送信される.その後,管理者から申請に対して承認されると事業者会員へとなることができる.また,事業者会員は会員自身が投稿したピンに表示されるアイコンを設定することができる.
管理者への問い合わせを行いたい場合はホーム画面から「お問い合わせ」を選択すると,問い合わせ画面が表示される.件名と管理者に問い合わせたい質問や要望を記入し,「送信する」を選択すると管理者へ送信される.問い合わせへの管理者からの返答は登録されているメールアドレスに返信される.

事業者会員が自身が行った投稿についての情報を閲覧したい場合は「ダッシュボード」を選択し,「概要」タブを選択することで閲覧することができる.「概要」タブでは総投稿数・総リアクション数・総閲覧数・エンゲージメント率や直近1週間のリアクション数と閲覧数の推移,ジャンル別の投稿数を確認することができる.また,リアクション数が多い上位5つの投稿が人気投稿として表示される.また,事業者会員が現在の契約状況や事業者登録料を確認したい場合は「ダッシュボード」から「支払い情報」を選択することで閲覧することができる.「支払い情報」では現在のプランや月額の料金,次回の請求日や支払いの履歴を確認することができる.事業者プランを解約したくなった場合は,同画面から「事業者プランの解約」を選択することで解約を行うことができる.