\subsection{会員機能}
会員機能は,以下のような機能から構成されている.

\begin{itemize}
    \item 新規会員登録
    \item ログイン・ログアウト
    \item 退会
    \item 問い合わせ
    \item 事業者申請機能
\end{itemize}


\subsubsection{新規会員登録}

一般会員が新規会員登録として,Googleアカウントを用いて会員になる機能.

利用規約に同意し,「Googleでログイン」を選択すると,Google認証に移る.
次に,登録するアカウントを選択,そのアカウント情報をDBに格納する.
これでログインが完了し,ホーム画面に遷移する.

\begin{figure}[H]
        \centering
        \includegraphics[scale=0.5]{sections/fig_account_general/account_gen1.jpg}
        \caption{新規会員機能}

        %\label{}

\end{figure}

\subsubsection{ログイン・ログアウト}
一般会員がGoogle認証を用いて、任意にログイン・ログアウトを行う機能.

ログイン機能では,一般会員が Google アカウントを用いた認証によってシステムにログインする.
ログインの処理は新規会員登録時の認証手順とほぼ同様であり,ログイン認証画面で取得したアカウント情報が
データベースに登録済みである場合にログインを許可する.

\begin{figure}[H]
    \centering
    \includegraphics[scale=0.5]{sections/fig_account_general/account_gen2_1.jpg}
    \caption{ログイン}
    %\label{fig:pinpost_bus1_1}
\end{figure}

ログアウト機能では,すでにログインしている会員が任意のタイミングでログアウト操作を行うことができる.
ログアウト操作時にはまずログアウト画面を表示し,ユーザがキャンセルを選択した場合はホーム画面へ戻る.
ログアウトを確定した場合はセッション情報を無効化し,ログイン画面へ遷移する.


\begin{figure}[H]
    \centering
    \includegraphics[scale=0.5]{sections/fig_account_general/account_gen2_2.jpg}
    \caption{ログアウト}
    %\label{fig:pinpost_bus1_1}
\end{figure}

\subsubsection{退会}
退会機能では,会員が任意のタイミングで本システムから退会できる.退会した会員の情報は,退会処理が完了した時点で完全に消去する.

退会の流れとしては,まず「マイページ」をクリックしてマイページ画面に遷移し,設定を選択すると退会画面が表示される.
続いてアカウント削除画面を表示し,ユーザは退会理由(任意)を入力し,削除確認項目にすべてチェックを入れる.
この状態で「アカウントを削除する」を選択するが,キャンセルを選択した場合はマイページ画面へ戻る.

そのまま「アカウントを削除する」を選択した場合は,「本当にアカウントを削除してもよろしいですか?この操作は取り消せません.」という確認画面を表示する.
ここでキャンセルを選択した場合はアカウント削除画面に戻り,「OK」を選択した場合はデータベースにアクセスして会員情報を削除し,ログイン画面へ遷移する.

\begin{figure}[H]
    \centering
    \includegraphics[scale=0.5]{sections/fig_account_general/account_gen3.jpg}
    \caption{退会}
    %\label{fig:pinpost_bus1_1}
\end{figure}

\subsubsection{問い合わせ}
問い合わせ機能では,一般会員が質問や要望を運営側に送信することができる.
質問や要望はメール形式で送信され,運営側からの回答は,会員が登録している Google アカウントのメールアドレスへ送信される.

問い合わせの流れとしては,「お問い合わせ」を選択すると問い合わせ画面が表示される.
ユーザは件名とメッセージを入力し,「送信する」を選択する.件名またはメッセージのいずれかが未記入の場合は,「このフィールドを入力してください.」と表示する.
両方が入力されている場合は,データベースにアクセスして件名およびメッセージを保存し,「お問い合わせを送信しました.運営からの返信をお待ちください.」と表示する.

\begin{figure}[H]
    \centering
    \includegraphics[scale=0.5]{sections/fig_account_general/account_gen4.jpg}
    \caption{問い合わせ}
    %\label{fig:pinpost_bus1_1}
\end{figure}

\subsubsection{事業者申請機能}
事業者登録申請機能は,一般会員が事業者会員への昇格を運営側へ申請するための機能である.
会員は店舗名,電話番号,住所などの情報を入力し,運営側による承認を経て事業者会員となる.

申請の流れとしては,まず「マイページ」をクリックしてマイページ画面を表示し,「事業者登録を申請」を選択する.
次に事業者登録申請画面が表示され,ユーザは店舗名,電話番号,住所を入力する.ここで「申請」を選択せずにキャンセルを選択した場合は,マイページ画面へ戻る.

「申請」を選択した場合は,データベースにアクセスして事業者情報を保存し,「事業者登録申請を送信しました.運営からの承認をお待ちください.」と表示する.

\begin{figure}[H]
    \centering
    \includegraphics[scale=0.5]{sections/fig_account_general/account_gen5.jpg}
    \caption{事業者申請機能}
    %\label{fig:pinpost_bus1_1}
\end{figure}


