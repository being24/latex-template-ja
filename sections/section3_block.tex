\subsection{ブロック機能}
ブロック機能は,一般会員が特定の一般会員に対して表示制限を行うための機能である.
本機能を利用することで,一般会員は不快・迷惑と感じる相手の投稿やリアクションを非表示にし,
快適な利用環境を維持することができる.

ブロック機能は以下の要素から構成される.

\begin{itemize}
  \item ブロック操作機能
  \item ブロック解除機能
\end{itemize}

\subsubsection{ブロック操作機能}
一般会員は相手のプロフィール画面または投稿詳細画面から「ブロック」ボタンを選択し,
対象一般会員のブロック操作を行うことができる.

ブロック操作を実行すると確認ダイアログが表示され,一般会員が意図した操作であるかを確認する.
確定後,ブロック対象の一般会員IDがブロックリストに登録され,システムが非表示処理を開始する.
以下のシステム構成図では,「OK」を選択する分岐において,キャンセルを押した場合は「NO」へ分岐するものとする.

\begin{figure}[H]
    \centering
    \includesvg[inkscapelatex=false, width=1.1\linewidth]{sections/fig_blockpintuho/block.svg}
    \caption{地図上でのピン表示機能}
    \label{fig:pin_main1}
\end{figure}

\subsubsection{非表示処理機能}
ブロックされた一般会員に関する情報を一般会員の画面上に表示しない機能である.
非表示となる情報は以下の通りである.

\begin{itemize}
  \item ブロック対象一般会員の投稿(地図上のピン・投稿一覧)
\end{itemize}

これにより,ブロック対象一般会員が投稿したピンは地図上に描画されない.

\begin{figure}[H]
    \centering
    \includesvg[inkscapelatex=false, width=1.1\linewidth]{sections/fig_blockpintuho/block2.svg}
    \caption{地図上でのピン表示機能}
    \label{fig:pin_main1}
\end{figure}

\subsubsection{ブロック解除機能}
一般会員は設定画面内の「ブロックリスト」から,過去にブロックした一般会員を確認し,
必要に応じてブロックを解除することができる.

ブロックを解除すると,対象一般会員の投稿やリアクションが再び表示されるようになる.
解除操作も誤操作を防ぐために確認ダイアログを表示し,一般会員の意図を確認してから処理を行う.

\begin{figure}[H]
    \centering
    \includesvg[inkscapelatex=false, width=1.1\linewidth]{sections/fig_blockpintuho/unblock.svg}
    \caption{地図上でのピン表示機能}
    \label{fig:pin_main1}
\end{figure}
