\subsection{並び替え機能}
並び替え機能は,以下のような機能から構成されている.

\begin{itemize}
    \item 距離順
    \item リアクション数順
    \item 新着順
\end{itemize}

\subsubsection{距離順}

距離順では,現在地または任意の場所からの距離が近い順にピンを並べ替えて表示する.

操作の流れとしては,まず並び替え機能をクリックし,並び替え項目を展開して表示する.
その後,「距離順」を選択すると,データベースにアクセスし,投稿内容を取得して距離順に並び替え,画面に表示する.

\begin{figure}[H]
        \centering
        \includegraphics[scale=0.5]{sections/fig_sort_general/sort_gen1.jpg}
        \caption{距離順}

        %\label{}

\end{figure}

\subsubsection{リアクション数順}

リアクション数順では,リアクションの数が多い順または少ない順にピンの表示を並べ替えて表示する。

操作の流れとしては,まず並び替え機能をクリックし,並び替え項目を展開して表示する.
その後,「リアクション数順」を選択すると,データベースにアクセスし,投稿内容を取得してリアクション数順に並び替え,画面に表示する.

\begin{figure}[H]
        \centering
        \includegraphics[scale=0.5]{sections/fig_sort_general/sort_gen2.jpg}
        \caption{リアクション数順}

        %\label{}

\end{figure}

\subsubsection{新着順}

新着順では,投稿日時が新しい順にピンの表示を並び替えて表示する。

操作の流れとしては,まず並び替え機能をクリックし,並び替え項目を展開して表示する.
その後,「新着順」を選択すると,データベースにアクセスし,投稿内容を取得して新着順に並び替え,画面に表示する.

\begin{figure}[H]
        \centering
        \includegraphics[scale=0.5]{sections/fig_sort_general/sort_gen3.jpg}
        \caption{新着順}

        %\label{}

\end{figure}