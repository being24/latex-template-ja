\subsection{通報機能}
SNS通報機能は,ユーザが不適切な投稿を運営へ通知するための機能である.
本機能は以下の要素から構成される.

\begin{itemize}
  \item 通報ボタン表示機能
  \item 通報内容入力機能
  \item 通報送信機能
\end{itemize}

\subsubsection{通報ボタン表示機能}
各投稿の詳細画面において「通報」ボタンが表示される.
ユーザが不適切と感じた投稿に対して,ワンタップで通報画面に遷移できるように設置されている.
投稿閲覧中に常に同じ位置に表示されるため,ユーザは迷うことなく通報操作を開始することができる.


\subsubsection{通報内容入力機能}
通報ボタンを押すと通報入力画面が表示される.
通報入力画面では,以下の情報をユーザが選択または入力できる.

\begin{itemize}
  \item 通報理由(暴言,不適切画像,スパム等)
  \item 具体的な説明(任意)
  \item 通報対象投稿の概要(自動表示)
\end{itemize}

ユーザが入力した内容に応じて,通報理由とともに投稿が運営に送信される.
入力フォームの上には通報対象の投稿が簡易表示され,ユーザが誤って別の投稿を通報することを防ぐ.


\subsubsection{通報送信機能}
「通報する」ボタンを押すことで通報内容が運営サーバへ送信され,通報受付完了メッセージを表示される.

通報が成功すると投稿画面に戻ることでユーザは処理完了を確認できる.

通報内容が入力されていない場合,通報理由がない場合は「通報理由を入力してください」と表示される.

\begin{figure}[H]
        \centering
        \includesvg[inkscapelatex=false, width=1\linewidth]{sections/fig_blockpintuho/tuho.svg}
        \caption{通報機能}
        \label{fig:report3}
\end{figure}

\begin{figure}[H]
        \centering
        \includesvg[inkscapelatex=false, width=1\linewidth]{sections/fig_blockpintuho/tuho2.svg}
        \caption{通報機能}
        \label{fig:report3}
\end{figure}

