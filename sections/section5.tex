\section{管理者側サブシステム構成}
管理者側の機能は以下のような機能から構成されている.

\begin{itemize}
    \item 投稿削除機能
    \item アカウント削除機能
    \item 事業者アカウントへの変更機能
    \item 問い合わせへの対応機能
    \item 運用状況モニタリング機能
    \item 通報機能
\end{itemize}

\subsection{投稿削除機能}
投稿削除タブを選択すると,投稿一覧を格納しているデータベースにアクセスし,その一覧を表示する.その一覧の右にあるゴミ箱マークを押すことで該当する投稿をデータベースから削除することができる.また,投稿したアカウントに削除した旨の通知を送る.最後に,「削除しました」というポップが表示される.

\begin{figure}[H]
    \centering
    \includesvg[keepaspectratio, width=0.8\linewidth]{figures/delete_post.svg}
    \caption{投稿削除機能}
    \label{fig:delete_post}
\end{figure}

\subsection{アカウント削除機能}

ユーザー管理タブを選択すると,ユーザー一覧を格納しているデータベースにアクセスし,ユーザー一覧を表示する.右側にあるゴミ箱ボタンを押すことで該当する利用者とその投稿とリアクションを全て削除する.削除後に,「削除しました」というポップが表示される.

\begin{figure}[H]
    \centering
    \includesvg[keepaspectratio, width=0.8\linewidth]{figures/delete_user.svg}
    \caption{アカウント削除機能}
    \label{fig:delete_user}
\end{figure}

\subsection{事業者アカウントへの変更機能}
事業者申請タブを選択すると,申請情報を格納しているデータベースにアクセスし,申請情報一覧を表示する.事業者申請を却下する場合,申請情報をデータベースから削除し,「却下しました」というポップが表示される.承認する場合,事業者会員情報を格納しているデータベースに新しくタプルを作成し,元々の会員情報と紐付けを行う.紐付け後,「承認しました!」というポップを表示する.

\begin{figure}[H]
    \centering
    \includesvg[keepaspectratio, width=0.8\linewidth]{figures/business_account.svg}
    \caption{事業者アカウントへの変更機能}
    \label{fig:business_account}
\end{figure}

\subsection{問い合わせへの対応機能}

問い合わせタブを選択すると,問い合わせ情報を格納しているデータベースにアクセスし,未対応の問い合わせ情報のみを表示する.そこから返信ボタンを押し,返信をすることでそれを利用者に送信すると共に該当の問い合わせを対応済みに変更し,「対応済みにしました」というポップを表示する.

\begin{figure}[H]
    \centering
    \includesvg[keepaspectratio, width=0.8\linewidth]{figures/support_request.svg}
    \caption{問い合わせへの対応}
    \label{fig:support_request}
\end{figure}
\subsection{運用状況モニタリング機能}

概要タブを選択すると,以下の要素をデータベースから参照し,表示する.
\begin{itemize}
    \item 総ユーザー数
    \item アクティブユーザー数
    \item 総投稿数
    \item 事業者アカウント数
    \item 未処理通報数
    \item 1週間の新規投稿数
    \item 1週間のリアクション数
    \item ジャンルのカテゴリー分布
\end{itemize}

\begin{figure}[H]
    \centering
    \includesvg[keepaspectratio, width=0.8\linewidth]{figures/monitoring.svg}
    \caption{運用状況モニタリング機能}
    \label{fig:monitoring}
\end{figure}

\subsection{通報機能}
通報管理タブを選択すると,通報一覧情報を格納しているデータベースにアクセスし,未処理の通報のみを表示する.
その通報を却下する場合は,該当通報を処理済みに変更し,「処理しました」というポップを表示する.承諾する場合は,該当投稿とそのリアクションを削除し,「削除しました!」というポップを表示する.

\begin{figure}[H]
    \centering
    \includesvg[keepaspectratio, width=0.8\linewidth]{figures/report.svg}
    \caption{通報機能}
    \label{fig:report}
\end{figure}