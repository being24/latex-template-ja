\subsection{保守・メンテナンス}
システムの継続的な改善と安定稼働のため,潜在的な問題を早期発見するために月次でシステムの健全性をチェックする.また,運用時に貯めたログ等が肥大化しないようにするために,ローテーションや圧縮を行う.また,アクセス状況やデータベースのクエリパフォーマンスを分析し,最適化を実施する.


\subsection{障害対応}
障害発生時の迅速な対応を可能にするために,監視システムを導入,障害を自動検知し,管理者に通知される仕組みにする.また,想定される障害シナリオごとに対応手順書を作成,迅速な復旧を可能とする.障害発生後は,原因分析を行うことで再発防止を目指す.
