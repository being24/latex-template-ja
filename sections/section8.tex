\section{データベース設計}
本システムのデータベース設計を以下に示す.図 \ref{fig:sec8/er} はデータの関係をER図で表したものであり,表 \ref{tb:user} から表 \ref{tb:reaction}はデータベースのテーブル設計である.
\begin{figure}[H]
    \centering
    \includesvg[inkscapelatex=false, width=1.1\linewidth]{sections/section8_database.svg}
    \caption{ER図}
    \label{fig:sec8/er}
\end{figure}

\if0
\begin{table}[H]
    \centering
    \caption{}
    \label{tb:}
    \begin{tabular}{|c|c|c|c|c|c|} \hline
        カラム名 & PK & FK & その他の制約 & 論理データ型 & 説明 \\ \hline\hline
    \end{tabular}
\end{table}
\fi

\begin{table}[H]
    \centering
    \caption{会員情報}
    \label{tb:user}
    \begin{tabular}{|c|c|c|c|c|c|} \hline
        カラム名 & PK & FK & その他の制約 & 論理データ型 & 説明 \\ \hline\hline
        id & ○ & & not null & VARCHAR(50) & Google ID \\ \hline
        gmail & & & not null & VARCHAR(100) & メールアドレス \\ \hline
        role & & & not null & ENUM('user','business','admin') & 会員区分 \\ \hline
        businessId & & ○ & not null & INT & 事業者ID \\ \hline
    \end{tabular}
\end{table}

\begin{table}[H]
    \centering
    \caption{事業者会員情報}
    \label{tb:business}
    \begin{tabular}{|c|c|c|c|c|c|} \hline
        カラム名 & PK & FK & その他の制約 & 論理データ型 & 説明 \\ \hline\hline
        id & ○ & & not null & INT & ID \\ \hline
        businessName & & & not null & VARCHAR(50) & 事業者名 \\ \hline
        kanaBusinessName & & & not null & VARCHAR(50) & カナ事業者名 \\ \hline
        zipCode & & & not null & INT & 郵便番号 \\ \hline
        address & & & not null & VARCHAR(100) & 住所 \\ \hline
        phone & & & not null & INT & 電話番号 \\ \hline
        registDate & & & not null & DATETIME & 登録日時 \\ \hline
        profileImage & & & & BLOB & プロフィール画像 \\ \hline
        userId & & ○ & not null & VARCHAR(50) & Google ID \\ \hline
    \end{tabular}
\end{table}

\begin{table}[H]
    \centering
    \caption{支払い情報}
    \label{tb:pay}
    \begin{tabular}{|c|c|c|c|c|c|} \hline
        カラム名 & PK & FK & その他の制約 & 論理データ型 & 説明 \\ \hline\hline
        id & ○ & & not null & INT & ID \\ \hline
        businessId & & ○ & not null & INT & 事業者ID \\ \hline
        payAmount & & & not null & INT & 支払額 \\ \hline
        date & & & not null & DATETIME & 支払日 \\ \hline
        payFlag & & & not null & BOOLEAN & 支払いフラグ \\ \hline
    \end{tabular}
\end{table}

\begin{table}[H]
    \centering
    \caption{ブロック情報}
    \label{tb:block}
    \begin{tabular}{|c|c|c|c|c|c|} \hline
        カラム名 & PK & FK & その他の制約 & 論理データ型 & 説明 \\ \hline\hline
        id & ○ & & not null & INT & ID \\ \hline
        blockerId & & ○ & not null & VARCHAR(50) & ブロックしたユーザーのGoogle ID \\ \hline
        blockedId & & ○ & not null & VARCHAR(50) & ブロックされたユーザーのGoogle ID \\ \hline
    \end{tabular}
\end{table}

\begin{table}[H]
    \centering
    \caption{問い合わせ情報}
    \label{tb:ask}
    \begin{tabular}{|c|c|c|c|c|c|} \hline
        カラム名 & PK & FK & その他の制約 & 論理データ型 & 説明 \\ \hline\hline
        id & ○ & & not null & INT & ID \\ \hline
        date & & & not null & DATETIME & 問い合わせ日時 \\ \hline
        subject & & & not null & VARCHAR(100) & 件名 \\ \hline
        text & & & not null & TEXT & 本文 \\ \hline
        askId & & ○ & not null & VARCHAR(50) & 問い合わせたユーザーのGoogle ID \\ \hline
        askFlag & & & not null & BOOLEAN & 問い合わせフラグ \\ \hline
        query & & & & TEXT & 下書き \\ \hline
    \end{tabular}
\end{table}

\begin{table}[H]
    \centering
    \caption{投稿内容}
    \label{tb:post}
    \begin{tabular}{|c|c|c|c|c|c|} \hline
        カラム名 & PK & FK & その他の制約 & 論理データ型 & 説明 \\ \hline\hline
        id & ○ & & not null & INT & ID \\ \hline
        placeId & & ○ & not null & INT & 場所ID \\ \hline
        userId & & ○ & not null & VARCHAR(50) & 投稿者ID \\ \hline
        postDate & & & not null & DATETIME & 投稿日時 \\ \hline
        text & & & not null & TEXT & 投稿の説明文 \\ \hline
        postImage & & & & BLOB & 投稿画像 \\ \hline
        numReaction & & & not null & INT & リアクション数 \\ \hline
    \end{tabular}
\end{table}

\begin{table}[H]
    \centering
    \caption{場所情報}
    \label{tb:place}
    \begin{tabular}{|c|c|c|c|c|c|} \hline
        カラム名 & PK & FK & その他の制約 & 論理データ型 & 説明 \\ \hline\hline
        placeId & ○ & & not null & INT & 場所ID \\ \hline
        postId & ○ & ○ & not null & INT & 投稿ID \\ \hline
        numPost & & & not null & INT & 投稿数 \\ \hline
        latitude & & & not null & DOUBLE(1, 4) & 緯度 \\ \hline
        longitude & & & not null & DOUBLE(1, 4) & 経度 \\ \hline
    \end{tabular}
\end{table}

\begin{table}[H]
    \centering
    \caption{ジャンル情報}
    \label{tb:genre}
    \begin{tabular}{|c|c|c|c|c|c|} \hline
        カラム名 & PK & FK & その他の制約 & 論理データ型 & 説明 \\ \hline\hline
        genreId & ○ & & not null & INT & ジャンルID \\ \hline
        postId & ○ & ○ & not null & INT & 投稿ID \\ \hline
        genreName & & & not null & 
        \begin{tabular}{c}
            ENUM('food', 'event', 'scene', \\'store', 'emergency', 'other')
        \end{tabular}
        & ジャンル名 \\ \hline
        color & & & not null & VARCHAR(6) & 表示色 \\ \hline
    \end{tabular}
\end{table}

\begin{table}[H]
    \centering
    \caption{通報情報}
    \label{tb:report}
    \begin{tabular}{|c|c|c|c|c|c|} \hline
        カラム名 & PK & FK & その他の制約 & 論理データ型 & 説明 \\ \hline\hline
        id & ○ & & not null & INT & ID \\ \hline
        userId & & ○ & not null & VARCHAR(50) & 通報したユーザーのGoogle ID \\ \hline
        postId & & ○ & not null & INT & 対象投稿ID \\ \hline
        reason & & & not null & TEXT & 通報理由 \\ \hline
        date & & & not null & DATETIME & 通報日時 \\ \hline
        reportFlag & & & not null & BOOLEAN & 対応済みフラグ \\ \hline
    \end{tabular}
\end{table}

\begin{table}[H]
    \centering
    \caption{事業者申請情報}
    \label{tb:businessReq}
    \begin{tabular}{|c|c|c|c|c|c|} \hline
        カラム名 & PK & FK & その他の制約 & 論理データ型 & 説明 \\ \hline\hline
        id & ○ & & not null & INT & ID \\ \hline
        name & & & not null & VARCHAR(50) & 事業者名 \\ \hline
        address & & & not null & VARCHAR(100) & 住所 \\ \hline
        phone & & & not null & INT & 電話番号 \\ \hline
        userId & & ○ & not null & VARCHAR(50) & 申請者のGoogle ID \\ \hline
    \end{tabular}
\end{table}

\begin{table}[H]
    \centering
    \caption{リアクション情報}
    \label{tb:reaction}
    \begin{tabular}{|c|c|c|c|c|c|} \hline
        カラム名 & PK & FK & その他の制約 & 論理データ型 & 説明 \\ \hline\hline
        id & ○ & & not null & INT & ID \\ \hline
        userId & ○ & ○ & not null & VARCHAR(50) & リアクションをしたユーザーのGoogle ID \\ \hline
        postId & & ○ & not null & INT & 投稿ID \\ \hline
    \end{tabular}
\end{table}